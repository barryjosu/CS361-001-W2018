\documentclass[12pt]{article}
%\usepackage{times}
\usepackage{hyperref}
\usepackage{color}
%this is a comment
\title{Assignment 1 - Vision Statement - Stock Trading Simulation}
\author{written by \\* James Barry - barryj \\* Eric Sisson - sissone}

\begin{document}
\maketitle
\pagebreak
\tableofcontents
\pagebreak

\section{Introduction}

A stock is a share in the ownership of a company. Owning a stock represents a claim on a percentage of the assets and earnings of the respective company for that stock. Stock trading is the act of buying and selling stocks with the goal of earning a personal profit. This is done on the stock market through a system of stock brokers, who help traders find buyers and sellers of stocks. Stock trading often involves large sums of money, scaring many away from ever partaking in it. A safe and simulated way to learn and practice stock trading would benefit anyone looking to join the world of investing. This proposal will cover possible methods of implementing a simulated stock trading environment, as well as the possible risks and advantages of creating such an application. 

\section{Summary}

Since stock trading is risky, a game environment to learn how trading works could help people wanting to try trading themselves. It's difficult to get any practical experience in trading without losing money, as explained in The Motley Fool's \color{blue}\underline{\href{https://www.fool.com/investing/general/2004/07/09/the-stock-market-is-risky.aspx}{The Stock Market is Risky (2004)}} \color{black}. There are plenty of articles, classes, and \color{blue}\underline{\href{https://www.investopedia.com/university/stocks/}{how-to guides}} \color{black} to learn how it all works, but one can't actually try stock trading without spending some money. This is the inspiration behind our proposal. Using this simulation to practice trading is intended to be a part of the process of learning trading, after one has learned about how it works elsewhere.
\\ \\
There are three key players in stock trading: the trader, the broker, and the market. The trader is the person looking to buy and sell stocks, the broker is the person that helps to fulfill the buying and selling of stocks, and the market represents all the other traders and brokers in the world, as well as the companies whose stocks are being traded. These are the three bodies that would need to be developed for a simulated environment.

\section{Implementation}

In designing a simulated trading environment, our high-level goal is to create a safe environment for a user to get risk-free experience in stock trading. We will utilize object oriented programming in Java to create a state for the user and market, keeping track of data like amount of money the user has, amount of stocks the user has, what stocks the user has, and current stock prices on the market. We will store this data in a save file,allowing the user to close the program and later pick up where they left off. The user will also be able to chose how large their starting budget is when they start a new game. The user will be able to act with a virtual broker to buy and sell stocks; to simplify the interaction, the broker will simply instantly fulfill buying and selling. In reality, a broker goes through a negotiation process on behalf of a trader, but this does not need to be included in a trader-oriented simulation. 
\\ \\
Simulating a realistic market will be the greatest roadblock in implementing this system. Mirroring all the nuances of the real-world stock market would be extremely difficult, considering time constraints and the potential knowledge of our development team. A potential solution to this is to implement an API for real-time stock market prices, such as \color{blue}\underline{\href{https://www.alphavantage.co/}{AlphaVantage}} \color{black}, but this means that the user will not affect the market since they are trading in simulated stocks. However, it would also mean that the simulated market would be less misleading, as a poorly simulated market could give the user a poor impression of what the stock market is actually like.

\section{Risks and Limitations}

Our implementation is mainly limited in its potential accuracy. If we chose to create a fake stock market, we run the risk of the simulation going awry and giving the user a faulty impression of how the market actually works. Using an API for real-time market values would fix this issue, but it also means the user would not be able to directly affect the market, making them feel like a ghost of sorts. Misleading the user would likely be a bigger risk, however, since the intention of this simulation is to help them develop the skills needed to trade real stocks for real money. Trusting a faulty simulation could lead a user to lose considerable money on the real market. 
\\ \\
We do not intend include any lessons on how the market actually works in our simulation. The user will simply be given a budget and left to their own devices. This simulation is meant to be an extension on the learning process, meaning that the user should learn how the market works through articles or relevant classes. A user should not expect this program to teach them how the stock market runs.
\\ \\
A limitation of implementing a fake stock market rather than using an API would be the difficulty of simulating market events. Active trading would not be practical in our simulation, as the user would not be able to watch the news and follow corporate events in an attempt to predict how the market will change. This limitation supports the use of an API.
\\ \\
Based on the reasons covered above, we will likely implement the simulation using an API rather than a fake stock market. This will help create accuracy and prevent the simulation from misleading the user. 

\end{document}